\documentclass[10pt]{beamer}


\usepackage[utf8]{inputenc}
\usetheme{metropolis}
\usepackage{appendixnumberbeamer}
\usepackage{pgfgantt}


\usepackage{booktabs}
\usepackage[scale=2]{ccicons}

\usepackage{pgfplots}
\usepgfplotslibrary{dateplot}

\usepackage{xspace}
\newcommand{\themename}{\textbf{\textsc{metropolis}}\xspace}


\title{High Level Assembler Plugin}
\subtitle{}
\date{Supervisor: Miroslav Kratochvíl}
\author{Michal Bali, Marcel Hruška, Peter Polák, Adam Šmelko, Lucia Tódová}
% \institute{Center for modern beamer themes}
% \titlegraphic{\hfill\includegraphics[height=1.5cm]{logo.pdf}}

\begin{document}

\maketitle

\begin{frame}{Table of contents}
  \setbeamertemplate{section in toc}[sections numbered]
  \tableofcontents[hideallsubsections]
\end{frame}

\section{Introduction}

\begin{frame}[fragile]{Metropolis}

    Maybe:
    
    Motivation:
    
    \includegraphics[width=10cm]{img/maxresdefault}
    
  
\end{frame}

\begin{frame}[fragile]{The HLASM language}

There are 3 types of instructions:
\begin{enumerate}
	\item Machine instructions
	\item Assembler instructions
	\item Conditional assembly instructions
\end{enumerate}


*Short slide just to introduce instructions*

\end{frame}


\begin{frame}[fragile]{Ordinary assembly example}
  
	\begin{verbatim}
	[00]              LR    1,2
	[01]              DS    CL(LEN)
	[02]   ADDR       DS    CL(SIZE)
	[03]
	[04]   HERE       DS    0H
	[05]   LEN        EQU   HERE-ADDR
	[06]   SIZE       EQU   1
	\end{verbatim}
	
	*Drawing of resulting object*
	
	*Explain the example* 
\end{frame}

\begin{frame}[fragile]{Conditional assembly example}


\begin{verbatim}
[01]     *CA example*

\end{verbatim}


\end{frame}

\begin{frame}[fragile]{Architecture}
\centering
\includegraphics[width=10cm]{img/hlasm_architecture}

\end{frame}



\begin{frame}[fragile]{Preview version}

*Screen z preview verzie

\end{frame}


\begin{frame}[fragile]{Project timeline}



\end{frame}


\begin{frame}[standout]
  Questions?
\end{frame}

\end{document}
