\documentclass[10pt]{beamer}


\usepackage[utf8]{inputenc}
\usetheme{metropolis}
\usepackage{appendixnumberbeamer}
\usepackage[defaultsans]{droidsans}


\usepackage{tikz}
\usetikzlibrary{positioning}
\usepackage{fancyvrb}
\usepackage{booktabs}
\usepackage[scale=2]{ccicons}

\usepackage{xparse}

\usepackage{xspace}
\newcommand{\themename}{\textbf{\textsc{metropolis}}\xspace}

\usepackage{array}


\title{High Level Assembler Plugin}
%\subtitle{Software project}
\date{Supervisor: Miroslav Kratochvíl}
\author{Michal Bali, Marcel Hruška, Peter Polák, Adam Šmelko, Lucia Tódová}
% \institute{Center for modern beamer themes}
% \titlegraphic{\hfill\includegraphics[height=1.5cm]{logo.pdf}}

\begin{document}

\maketitle


\begin{frame}[standout]{Motivation}
\centering
\hspace*{-1cm}
\includegraphics[width=12.6cm]{img/maxresdefault}
\pause
\footnotesize Your Account Balance May Still Rely On Such Code!

\end{frame}

\begin{frame}{Project goals}
    Main goal: HLASM support for contemporary source code editors, using the Language Server Protocol
    \begin{itemize}
    	\item Code validation
    	\item Code navigation (Go to definition, Find all references, \dots)
    	\item Tooltips, autocompletion, \dots
    \end{itemize}

    In the long term:
    \begin{itemize}
        \item Simplify orientation in legacy code
        \item Allow transitioning to some modern language % COBOL
    \end{itemize}
\end{frame}


\begin{frame}{The HLASM language}

There are 3 types of instructions:
\begin{enumerate}
	\item Machine instructions
	\item Assembler instructions \\\quad{\small (`compile'-time variables, modifications of assembler program state)}
	\item Conditional Assembly instructions \\\quad{\small (basically a Turing-complete macro system)}
\end{enumerate}

\textbf{Main problem:} Code interpretation heavily depends on 2.~and 3.

\end{frame}

\newcommand{\aaa}[2]{\alert<#1>{#2}}
\begin{frame}[fragile]{Example --- Assembler instructions}
\begin{Verbatim}[commandchars=\\\{\}]
[00]              \aaa{2}{LR    1,2}
[01]              \aaa{3}{DS    CL(LEN)}
[02]   \aaa{4}{ADDR       DS    CL(SIZE)}
[03]   \aaa{5}{HERE       DS    CL5}
[04]
[05]   \aaa{6}{LEN        EQU   HERE-ADDR}
[06]   \aaa{7}{SIZE       EQU   1}
\end{Verbatim}

Output layout:
\begin{center}
\begin{tikzpicture}[node distance=0mm, font=\tt]
\node[rectangle,draw, minimum width=7em, minimum height=2em] (a) {\only<2->{\alert<2>{2}}};
\node[rectangle,draw, minimum width=7em, minimum height=2em, right=of a] (b) {\only<3->{\alert<3>{LEN}}\only<3-5>{=?}\only<6>{=\alert{SIZE}}\only<7>{=\alert{1}}};
\node[rectangle,draw, minimum width=7em, minimum height=2em, right=of b] (c) {\only<4->{\alert<4>{SIZE}}\only<4-6>{=?}\only<7>{=\alert{1}}};
\node[rectangle,draw, minimum width=7em, minimum height=2em, right=of c] (d) {\only<5->{\alert<5>{5}}};

\node[below=of b.south east, anchor=east, rotate=45] {\color<-3>{white}ADDR $\to$};
\node[below=of c.south east, anchor=east, rotate=45] {\color<-4>{white}HERE $\to$};
\end{tikzpicture}
\end{center}

\end{frame}

\begin{frame}[fragile]{Example --- Conditional Assembly}
\begin{Verbatim}[commandchars=\\\{\}]
[00]   &VAR       \aaa{3}{SETA    1}
[01]   .LOOP      ANOP         
[02]   \aaa{2}{SYM&VAR}    DC      C'A'
[03]              \aaa{4}{AIF     (D'SYM3).END}
[04]   \aaa{3}{&VAR       SETA    &VAR+1}
[05]              \aaa{3}{AIF     (&VAR LE 4).LOOP}
[06]   .END       END     
\end{Verbatim}

\pause
\textbf{Output:}
\begin{Verbatim}[commandchars=\\\{\}]
[00]   \aaa{2}{SYM1}       DC      C'A' \pause
[01]   SYM2       DC      C'A'
[02]   \aaa{4}{SYM3}       DC      C'A' \pause
[03]              END
\end{Verbatim}
\end{frame}

\begin{frame}{Architecture}
\centering
\includegraphics[width=10.5cm]{img/hlasm_architecture}
\end{frame}


\begin{frame}{Already done: Preview version}
\centering
\includegraphics[width=9cm]{img/screenshot}

\footnotesize
Source code taken from \url{https://github.com/zhaimlill/EnhanceStartOnZ}
\end{frame}


\begin{frame}{Current timeline}

\centering
\hspace*{-0.95cm}
\includegraphics[width=12.5cm]{img/timeline}

\end{frame}


\begin{frame}{Future perspectives}
\begin{itemize}
\item The prototype has been already tested\\ on realistically large projects
	\begin{itemize}
		\item ``proof of concept'' works well
		\item additional integration problems seem unlikely
	\end{itemize}
\item One possible problem:\\ parser performance
	\begin{itemize}
		\item small changes require reprocessing\\ of huge amounts of generated code
		\item optimizations are possible
	\end{itemize}
\end{itemize}
\end{frame}


\begin{frame}[standout]
  Thank you for attention.

  \small Questions?
\end{frame}


\end{document}
