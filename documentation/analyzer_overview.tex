\chapter {Analyzer}
\label{chap:analyzer}

The role of the analyzer is to provide a facade over objects and methods to create a simple interface for lexical and semantic processing (analyzing) of a single HLASM source file. The output of the analysis is the basic input of the LSP server.


After the analyzer is constructed, it analyzes the provided source file. As a result, it updates HLASM context tables and provides a list of diagnostics linked to the file, highlighting, list of symbol definitions, etc.


\sectionSrc{Overview}
{parser\_library/src/analyzer.h}

The analyzer is composed of several sub-components, all required to properly process the file. 
\begin{description}
	\item[LSP data collector] collects and retrieves all LSP information created while processing the file.
	\item[HLASM context tables] hold information about the context of the processed HLASM source code.
	\item[Lexer--Parser sub-components] simplify the processing interface and ease the use of this component. They are needed to create a source file parser.
	\item[Processing manager] executes the main loop where the file is processed.
\end{description}

LSP data collector is required by Lexer-Parser sub-components. They are composed into the parser object required by processing manager. HLASM context tables are used by the manager and the sub-components as well. 

The components together contribute to the proper functionality of the method \TT{analyze}. It processes a provided file and fills LSP data collector from which LSP information can be further retrieved.  

\subsection{Construction}

In order to parse a HLASM file, the analyzer class is constructed with the following parameters:
\begin{itemize}
	\item \emph{Name and content of a file.}
	\item \emph{Parse library provider} -- an object responsible for resolving source file dependencies. The dependencies are only discovered during the analysis, so it is not possible to provide the files beforehand.
	\item \emph{Processing tracer} (see \cref{macro_tracer}).
\end{itemize}

When this constructor is used, the analyzer creates HLASM context tables and processes the provided source as an open-code. We say that the analyzer has \emph{owner semantics}; it is the owner of the context tables. 
 
The analyzer provides \emph{reference semantics} as well (holding just a reference of the context tables). The provided source is not treated as an open-code, rather as an external file dependency. The constructor of an analyzer with reference semantics adds the following two parameters to the previous one:
\begin{itemize}
	\item \emph{HLASM context tables reference} --- belongs to the owning open-code analyzer.
	\item \emph{Library data} --- states how the dependency file should be treated (see \cref{lab06:lib_data}).
\end{itemize}

This constructor is called within open-code analyzer by its sub-components when they use the \emph{Parse library provider}.

