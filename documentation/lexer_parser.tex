\section{Lexer and Parser}
\label{lab06:parser}

Lexer's task is to read source string and break it into tokens --- small pieces of text with special meaning. The most important properties of the lexer:
\begin{itemize}
	\item each token has location in the source text
	\item has the ability to check whether all characters are valid in the HLASM source
	\item has the ability to jump in the source file backward and forward if necessary (for implementation of instructions like AGO and AIF). Because of this, it is not possible to use any standard lexing tool and the lexer has to be implemented from scratch.
\end{itemize}

\subsection{Parser}

Parser component takes the stream of tokens the lexer produces and recognizes HLASM statements according to the syntax. To accomplish this, a parser generator tool Antlr 4 \footnote{\url{https://www.antlr.org}} is used.

The input to Antlr is a grammar (written in antlr-specific language) that specifies the syntax of HLASM language and generates source code (in C++) for a recognizer, which is able to tell whether input source code is valid or not. Moreover, it is possible to assign a piece of code that executes every time a grammar rule is matched by the recognizer to further process the matched piece of code.
