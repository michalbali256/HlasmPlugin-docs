\section{LSP data collector}
\label{lsp_data}
The data collection is necessary to be able to reply to the LSP requests without the need to re-parse. During the parsing process, a component called \emph{LSP info processor} processes and stores this information. The main goal of this component is to collect as much information as possible to provide meaningful and complex replies to the LSP requests while maintaining the memory and parse-time overhead negligible.

\emph{LSP info processor} is invoked after each parsed and processed statement to collect and store the information it needs inside the \emph{LSP context} (part of HLASM context). 

We will describe the component in better detail as a cross product of distinct symbols and the LSP features. Possible LSP features are:
\begin{itemize}
	\item \emph{hover}
	\item \emph{complete}
	\item \emph{go\_to\_definition}
	\item \emph{references}
\end{itemize}
The symbols, on which the user might call mentioned LSP features, are:
\begin{itemize}
	\item \emph{instructions}
	\item \emph{variable symbols}
	\item \emph{sequence symbols}
	\item \emph{ordinary symbols}
\end{itemize}

The \emph{references} feature uses the same symbols as \emph{go\_to\_definition}, its usage is straightforward and therefore does nothing that needs to be further explained.

\subsection{Instructions}

The most complex symbols are instructions. To be able to provide information about each built-in instruction, we extract its type and parameters from the checker component (\cref{checker}).

As we know, user-defined macros are instructions as well. These are added to the list of all instructions after their first use in the open code. Macros can also be redefined in the open code, resulting in distinct macros with same name being used across the open code.

The \emph{hover} feature displays the instruction's type, the syntax of its parameters and, in case of macros, its version and documentation. The \emph{go\_to\_definition} works only in case of macros and goes to the definition of the macro.

Whenever a user types an A-Z character after arbitrary number of whitespaces from the left side, the \emph{complete} request for instructions is issued. This rolls out a list of all possible instructions with the information about them similar to the \emph{hover}.

\subsection{Variable Symbols}

Variable symbols are used in substitutions and begin with \& (\cref{var_sym}). Based on the instruction that has set the variable symbol, it may be of type \emph{number}, \emph{string} or \emph{bool}, which is displayed on both \emph{hover} and \emph{complete}. The variable symbols that are declared as macro parameters (and therefore have no type) are labeled as \emph{Macro Param}.

The \emph{complete} is triggered whenever user types \&. It responds with a list of variable symbols that have been defined so far.

The \emph{go\_to\_definition} jumps to the first occurrence of the symbol.

\subsection{Sequence Symbols}

Probably the most interesting information about the sequence symbols is their position, which is also the displayed information for both \emph{hover} and \emph{complete}.

The \emph{complete} is triggered whenever user types . and responds with list of sequence symbols that have been defined so far.

The \emph{go\_to\_definition} jumps to the target destination of the sequence symbol, i.e. the location where the code generation continues.

\subsection{Ordinary Symbols}

Ordinary symbols have the capability to hand over various information about the code. They are also the only symbols for which only the name placeholders with their positions are collected after each processed statement. After the parsing, these placeholders are matched with their values according to the ordinary processor (\cref{ord_proc}). \emph{COPY} instruction's parameter is also considered an ordinary symbol in the LSP context.  

The \emph{go\_to\_definition} either jumps to the definition of the ordinary symbol which is a label or, in case of a COPY parameter, to the file where the COPY is defined.

The \emph{hover} gives information whether the symbol is absolute or not, its value and its attributes' values. 




