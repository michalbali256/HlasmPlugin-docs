
\newcommand{\cool}{\color{green!50!white!80!black}}
\newcommand{\textcool}[1]{{\cool #1}}

\newcommand{\XX}[1]{\textcolor{red}{#1}}
\newcommand{\TT}[1]{\texttt{#1}}
\newcommand{\SC}[1]{{\fontfamily{phv}\selectfont\textsc{#1}}}

% Draw black "slugs" whenever a line overflows, so that we can spot it easily.


% avoid some slugs naturally
\clubpenalty=1000
\widowpenalty=1000
%\hyphenpenalty=100  % turn this on to prevent hyphenation
\emergencystretch=2cm


%%% The field of all real and natural numbers
\newcommand{\R}{\mathbb{R}}
\newcommand{\N}{\mathbb{N}}
\newcommand{\F}{\mathbb{F}}
\newcommand{\Z}{\mathbb{Z}}

\newcommand{\bms}{\begin{enumerate}[label=\bf (M\arabic*)]}
\newcommand{\bwp}{\begin{enumerate}[label=\bf \normalsize  (WP\arabic*), resume=del]}
\newcommand{\eenum}{\end{enumerate}}
\newcommand{\itemm}{\large \item }
\newcommand{\itemwp}{ \normalsize \item }
\newcommand{\deadline}[2]{\small (deadline: \textit{month #1}, duration: \textit{#2 moths})}
\newcommand{\people}[1]{\textit{\small (#1)}}


% move the headings out of gutenberg era
\setcounter{secnumdepth}{4}
\titleformat{\chapter}{\cool\fontsize{24pt}{24pt}\bfseries}{\color{black!25}\thechapter.}{1em}{}
\titleformat{\section}{\cool\fontsize{16pt}{18pt}\bfseries}{\scriptsize\color{black!25}\thesection}{1em}{}
\titleformat{\subsection}{\cool\fontsize{14pt}{16pt}\bfseries}{\scriptsize\color{black!25}\thesubsection}{1em}{}
\titleformat{\subsubsection}{\cool\fontsize{12pt}{14pt}\bfseries}{\scriptsize\color{black!25}\thesubsubsection}{1em}{}

% code floats
\colorlet{shadecolor}{cyan!10}
\makeatletter
\newcommand\floatc@code[2]{{\@fs@cfont #1} #2\par}
\newcommand\fs@code{\def\@fs@cfont{\bfseries}\let\@fs@capt\floatc@code
\def\@fs@pre{}%
\def\@fs@mid{\vspace{-.5ex}\begin{shaded}}%
\def\@fs@post{\vspace{-1em}\end{shaded}}%
\let\@fs@iftopcapt\iftrue}
\makeatother

\floatstyle{code}
\newfloat{listing}{tbp}{lst}
\floatname{listing}{Listing}

% file indexing
\renewcommand\indexname{Source file index}

\makeatletter
\def\patheach#1#2#3{\@test@patheachF{#1}{#2}#3/\endgroup}
\def\@test@patheachF#1#2#3\endgroup{\if#3\empty\empty\else\@patheachF{#1}{#2}#3\endgroup\fi}
\def\@patheachF#1#2#3/{#1{#3}\@test@patheach{#1}{#2}}
\def\@test@patheach#1#2#3\endgroup{\if#3\empty\empty\else\@patheach{#1}{#2}#3\endgroup\fi}
\def\@patheach#1#2#3/{#2#1{#3}\@test@patheach{#1}{#2}}
\def\foreach#1#2{\@test@foreach{#1}#2,\endgroup}
\def\@test@foreach#1#2\endgroup{\if#2\empty\empty\else\@foreach{#1}#2\endgroup\fi}
\def\@foreach#1#2,{#1{#2}\@test@foreach{#1}}
\makeatother

\def\makefileindex#1{%
\index{\patheach{\texttt}{!}{#1}}%
\hspace{-1ex}\texttt{\patheach{}{/\-}{#1}}\\}

\def\makefileindexes#1{\foreach{\makefileindex}{#1}}

\newcommand{\sectionFiles}[1]{ \strut~\hfill~\hspace{1ex}~\parbox{.3\linewidth}{\tt\scriptsize\textcolor{gray}{\makefileindexes{#1}}}}

\newcommand{\sectionSrc}[2]{\section[#1]{#1\sectionFiles{#2}}}
\newcommand{\subsectionSrc}[2]{\subsection[#1]{#1\sectionFiles{#2}}}
\newcommand{\subsubsectionSrc}[2]{\subsubsection[#1]{#1\sectionFiles{#2}}}
