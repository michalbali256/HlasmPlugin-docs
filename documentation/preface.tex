\chapter{Introduction}

The IBM High Level Assembler Language (HLASM) is still actively used commercially, even though it is a relatively old language. Its roots go back to the 1970s, when IBM made their first mainframes. Since then, the IBM assembler has been revised several times --- the last version (which is the concern of this project) was released in 1992. Although it is hard to believe, a lot of the software that has been written in the language over the years is still actively used and maintained, mainly because of the conservative mainframe users and IBM's vendor lock-in.

Today, HLASM developers are forced to code in archaic terminals directly on the mainframe. Therefore, they spend a lot of time navigating around the code and the environment. For example, solely due to the fact that the user needs to navigate through plenty of terminal screens it takes around a minute just to get to a screen where it is possible to make a change in a file and recompile. For developers, it would be extremely useful to have an IDE plugin that would minimize contact with the mainframe terminal, could analyze the HLASM program, check its validity and make the code clearer by syntax highlighting. 

We introduce such plugin that improves HLASM programming experience, so that it can be compared to coding in modern programming languages, by providing instant code validity checks, advanced highlighting, code analysis, and all the functionality that a programmer currently takes for granted when writing code.

Some of the most noteworthy properties and features of the plugin are:
\begin{itemize}
	\item It is capable of interpreting and tracing a large subset of HLASM code-generating instructions
	\item It contains a list of all built-in instructions that is used to validate the generated code
	\item It implements DAP and LSP protocols, providing interface to be easily integrated into numerous modern code editors
	\item It has been run and tested on over 15 millions lines of real production HLASM code
\end{itemize}

This document serves as an in-depth documentation for anyone who would like to understand the implementation of the project and the reasons behind it. It is advised that the potential contributors to the project read this documentation first.

\section{Organisation of this document}
First of all, in~\cref{hlasm}, we briefly explain the basics of HLASM needed to comprehend the workflow of this language. In~\cref{arch}, we provide an overview of the project's architecture, naming the most important components and indicating their relations. Then, we describe these components in separate chapters in further detail. In~\cref{chap:lang_server}, we state the responsibilities of the language server as the communication provider between the extension client and the parsing library. The workspace manager is the entry point to the parsing library used by the language server and it is fully described in~\cref{ws_manager}. The purpose of its sub-components is to handle file management, dependency resolution and parsing. The core of the processing of a HLASM file is implemented inside the analyzer, whose mechanics and implementation details are discussed in~\cref{chap:analyzer}. The project also provides macro tracing through the standard debugging procedure and it is fully explained in~\cref{macro_tracer}.The last mentioned component, detailed in~\cref{extension} is the VSCode extension, which communicates with the language server and provides IDE features to the user. At the end of this document, in~\cref{build}, we provide the instructions on how to build the project.
