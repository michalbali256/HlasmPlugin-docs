\section{External Macro Libraries and COPY Members}
\label{sec:configuration}

The HLASM Language Support extension looks for locally stored members when a macro or COPY instruction is evaluated. The paths of these members are specified in two configuration files in the \TT{.hlasmplugin} folder of the currently open workspace:
\begin{itemize}
	\item  \TT{proc\_grps.json} defines \textbf{processor groups} by assigning a group name to a list of directories. Hence, the group name serves as an unique identifier of a HLASM library set defined by a list of directories.
	
	
	\item  \TT{pgm\_conf.json} provides a mapping between \textbf{programs} (open-code files) and processor groups. It specifies which list of directories is used with which source file. If a relative source file path is specified, it is relative to the current workspace.
\end{itemize}

Therefore, when you want to use a predefined set of macro and copy members in your source code, you do: 
\begin{enumerate}
	\item Enumerate the library directories in \TT{proc\_grps.json} and name them with an identifier; thus, create a new processor group.
	\item Use the identifier of the new processor group with the name of your source code file in \TT{pgm\_conf.json} to assign the library members to the program.
\end{enumerate}

The structure of the configuration is based on CA Endevor® SCM.


\begin{listing}
	\begin{verbatim}
{
  "pgroups": [
    {
      "name":"GROUP1",
      "libs": [
        "ASMMAC/",
        "C:/SYS.ASMMAC"
      ]
    },
    {
      "name":"GROUP2",
      "libs": [
        "G2MAC/",
        "C:/SYS.ASMMAC"
      ]
    }
  ]
}
	\end{verbatim}
	\caption{This example defines two processor groups, GROUP1 and GROUP2, and a list of directories to search for macros and COPY files.}
	\label{lst:ex1}
\end{listing}

\begin{listing}
	\begin{verbatim}
{
  "pgms": [
    {
      "program": "source",
      "pgroup": "GROUP1"
    },
    {
      "program": "asm_prg",
      "pgroup": "GROUP2"
    },
  ]
}
	\end{verbatim}
	\caption{Example \TT{proc\_grps.json} specifies that GROUP1 is used when working with \TT{source\_code.hlasm} and GROUP2 is used when working with \TT{second\_file.hlasm}}
	\label{lst:grps}
\end{listing}

An example configuration can be seen in \cref{lst:ex1} and \cref{lst:grps}. It describes two processor groups \TT{GROUP1}, \TT{GROUP2} and their respective mappings to programs \TT{source} and \TT{asm\_prg}. If you have the two configuration files configured as above and invoke the MAC1 macro from \TT{source}, the folder \TT{ASMMAC/} in the current workspace is searched for a file named exactly MAC1. If that file is not found, the folder \TT{C:/SYS.ASMMAC} is searched. If that search is unsuccessful too, an error displays that the macro does not exist.

Note that the macro \TT{MAC1} is searched in directories in order as they are listed in the configuration. 

To add a possibility to specify a custom extension for HLASM files in the workspace, {pgm\_conf.json} has an optional field \TT{alwaysRecognize}. It takes an array of wildcards and provides the following functionality:
\begin{itemize}
	\item All files matching these wildcards will always be recognized as HLASM files. Hence, the HLASM plugin will be automatically activated on all the matched files.
	\item If an extension wildcard is defined, all macro and copy files with such extension may be used in the source code. For example, with the extension wildcard \TT{*.hlasm}, a user may add macro MAC to his source code even if it is in a file called \TT{Mac.hlasm}.
\end{itemize}

\Cref{lst:alw} demonstrates the mentioned functionality. With this configuration file, processor group \TT{GROUP1} will be assigned to \TT{source} and \TT{source.hlasm} file as well. Also, macro and copy files in the \TT{lib} directory can be referenced in the program without the \TT{.asm} extension  but still be correctly recognized.


\begin{listing}
	\begin{verbatim}
{
  "pgms": [
  {
    "program": "source",
    "pgroup": "GROUP1"
  }
  ],
   "alwaysRecognize" : ["*.hlasm", "libs/*.asm"]
}
	\end{verbatim}
	\caption{In this example, GROUP1 is used for all open code programs.}
	\label{lst:alw}
\end{listing}

Lastly, the program field in \TT{pgm\_conf.json} supports wildcards as well (see \cref{lst:wild}). This provides simplified use for users that share one processor group among all their programs.

\begin{listing}
	\begin{verbatim}
{
  "pgms": [
  {
    "program": "*",
    "pgroup": "GROUP1"  
  }
  ]
}
	\end{verbatim}
	\caption{In this example, GROUP1 is used for all open code programs.}
	\label{lst:wild}
\end{listing}
