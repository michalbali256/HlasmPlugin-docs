\section{Introduction}

The IBM High Level Assembler Language (HLASM) is still actively used commercially, even though it is a relatively old language. Its roots go back to the 1970s, when IBM made their first mainframes. Since then, the IBM assembler has been revised several times --- the last version (which is the concern of this project) was released in 1992. Although it is hard to believe, a lot of the software that has been written in the language over the years is still actively used and maintained, mainly because of the conservative mainframe users and IBM's vendor lock-in.

Today, HLASM developers are forced to code in archaic terminals directly on the mainframe. Therefore, they spend a lot of time navigating around the code and the environment. For example, solely due to the fact that the user needs to navigate through plenty of terminal screens it takes around a minute just to get to a screen where it is possible to make a change in a file and recompile.

In this document, we introduce HLASM Language Support. It is an extension that provides code completion, highlighting and navigation features, shows mistakes in the source, and lets you trace how the conditional assembly is evaluated with a modern debugging experience.

This extension is a part of the \href{https://github.com/eclipse/che-che4z}{Che4z} open-source project.

We organize this document as follows: \cref{sec:start} describes the installation of the plugin and its basic usage. In \cref{sec:features}, we review all features of the extension from the user perspective. Finally, \cref{sec:configuration} describes instructions for importing external macro libraries and using COPY members.