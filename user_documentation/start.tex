\section{Getting Started}
\label{sec:start}

\subsection{Installation}
There are two possible ways to install the HLASM Language Support extension: to download it from Visual Studio Marketplace\footnote{\url{https://marketplace.visualstudio.com/items?itemName=broadcomMFD.hlasm-language-support}}, or to build it from source and manually install it.

\begin{description}
	\item[Installation from Marketplace]
	The extension can be installed from the Visual Studio Code Marketplace by following these steps:
	\begin{enumerate}
		\item Open the extensions tab (Ctrl + Shift + X)
		\item Search for ``HLASM language support'' and select it.
		\item Select ``Install''.
	\end{enumerate}
	\item[Installation from source]
	The detailed guide describing how to build and install the extension from source is to be found in the Programmer's documentation.
\end{description}

\subsection{Usage}
To start using the HLASM Language Support extension, follow these steps:

\begin{enumerate}
	\item In \textbf{File} - \textbf{Open Folder...}, select the folder with your HLASM sources.
	\item Open your HLASM source code (no file extension is needed) or create a new file.
	\item If the extension fails to auto-detect HLASM language, set it manually in the bottom-right corner of the VS Code window.  
	\item The extension is now enabled on the opened file. If you have macro definitions in separate files or use the COPY instruction, proceed with the steps below to configure the extension to search for external files in the correct directories:
	\item After opening the HLASM file, two popups display. Select \TT{Create pgm\_conf.json with current program} and \TT{Create empty proc\_grps.json}. The two configuration files are created in the \emph{.hlasmplugin} subfolder.
	\item In the \emph{proc\_grps.json} file, fill the \emph{libs} array with paths to folders with macro definitions and COPY files. For example, if you have your macro files in the \emph{ASMMAC/} folder, type the string \emph{"ASMMAC"} into the libs array.
\end{enumerate}

There is an example workspace in the folder \TT{example\_workspace} that can be used to test out the extension.

For a full explanation of the configuration, see the~\cref{sec:configuration}.
