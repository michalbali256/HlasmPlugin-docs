\chapter{Getting Started}

To start using the HLASM Language Support extension, follow these steps:

\begin{enumerate}
	\item Install the extension either from Visual Studio Marketplace\footnote{\url{https://marketplace.visualstudio.com/items?itemName=broadcomMFD.hlasm-language-support}} or from source as described in chapter 10 of the project documentation.
	\item In \textbf{File} - \textbf{Open Folder...}, select the folder where your HLASM project is located.
	\item Open your HLASM source code (no file extension is needed) or create a new file.
	\item If the extension fails to auto-detect HLASM language, set it manually in the bottom-right corner of the VS Code window.  
	\item The extension is now enabled on the opened file. If you have macro definitions in separate files or use the COPY instruction, proceed with the steps below to configure the extension to search for external files in the correct directories:
	\item After opening the HLASM file, two popups display. Select \TT{Create pgm\_conf.json with current program} and \TT{Create empty proc\_grps.json}. The two configuration files are created in the \emph{.hlasmplugin} subfolder.
	\item In the \emph{proc\_grps.json} file, fill the \emph{libs} array with paths to folders with macro definitions and COPY files. For example, if you have your macro files in the \emph{ASMMAC/} folder, type the string \emph{"ASMMAC"} into the libs array.
\end{enumerate}

There is an example workspace in the folder \TT{example_workspace} that can be used to test out the extension.

For a full explanation of the configuration, see the~\cref{chap:configuration}.
