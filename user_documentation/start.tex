\section{Getting Started}
\label{sec:start}

\subsection{Installation}
There are two possible ways to install the HLASM Language Support extension: to download it from Visual Studio Marketplace\footnote{\url{https://marketplace.visualstudio.com/items?itemName=broadcomMFD.hlasm-language-support}}, or to build it from source and manually install it.

\paragraph*{Building from source}
To build the extension from source, you have to first install following prerequisites:

\begin{itemize}
	\item CMake 3.10 or higher
	\item C++ compiler with support for C++17
	\item Java Development Kit (JDK) 8 or higher
	\item Git
	\item npm
	\item Linux only: pkg-config
	\item Linux only: UUID library
	\item Mac only: LLVM 8
\end{itemize}

Type following commands:\\

Windows
\begin{listing}[H]
\begin{verbatim}
mkdir build && cd build
cmake ../
cmake --build .
\end{verbatim}
\end{listing}

{Ubuntu 18.04}
\begin{listing}[H]
\begin{verbatim}
apt update && sudo apt install cmake g++-8 uuid-dev npm default-jdk
pkg-config maven git
mkdir build && cd build
cmake -DCMAKE_C_COMPILER=gcc-8 -DCMAKE_CXX_COMPILER=g++-8 ../
cmake --build .
\end{verbatim}
\end{listing}

\pagebreak
{Alpine linux}
\begin{listing}[H]
\begin{verbatim}
apk update && apk add linux-headers git g++ cmake util-linux-dev npm
pkgconfig openjdk8 maven
mkdir build && cd build
cmake ../
cmake --build .
\end{verbatim}
\end{listing}

{MacOS 10.14}
\begin{listing}[H]
\begin{verbatim}
mkdir build && cd build
cmake -DCMAKE_C_COMPILER=clang -DCMAKE_CXX_COMPILER=clang++
-DLLVM_PATH=<path-to-llvm-installation> ../
cmake --build .
\end{verbatim}
\end{listing}

\paragraph*{Installation}
The built VSIX can be manually installed into VS Code by following these steps:
\begin{enumerate}
	\item Open the extensions tab (Ctrl + Shift + X)
	\item Select ``More actions ...'' (the $\cdots$ icon)
	\item Select ``Install from VSIX...''
	\item Find the VSIX file and confirm the selection.
	\item The plugin is now installed.
\end{enumerate}

Alternatively, the plugin can be installed with the following command:
\begin{listing}[H]
\begin{verbatim}
code --install-extension <path-to-vsix>
\end{verbatim}
\end{listing}

\subsection{Usage}
To start using the HLASM Language Support extension, follow these steps:

\begin{enumerate}
	\item In \textbf{File} - \textbf{Open Folder...}, select the folder where your HLASM project is located.
	\item Open your HLASM source code (no file extension is needed) or create a new file.
	\item If the extension fails to auto-detect HLASM language, set it manually in the bottom-right corner of the VS Code window.  
	\item The extension is now enabled on the opened file. If you have macro definitions in separate files or use the COPY instruction, proceed with the steps below to configure the extension to search for external files in the correct directories:
	\item After opening the HLASM file, two popups display. Select \TT{Create pgm\_conf.json with current program} and \TT{Create empty proc\_grps.json}. The two configuration files are created in the \emph{.hlasmplugin} subfolder.
	\item In the \emph{proc\_grps.json} file, fill the \emph{libs} array with paths to folders with macro definitions and COPY files. For example, if you have your macro files in the \emph{ASMMAC/} folder, type the string \emph{"ASMMAC"} into the libs array.
\end{enumerate}

There is an example workspace in the folder \TT{example\_workspace} that can be used to test out the extension.

For a full explanation of the configuration, see the~\cref{sec:configuration}.
